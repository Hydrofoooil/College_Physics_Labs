\documentclass[10pt,a4paper]{ctexart}
\usepackage[margin=2cm]{geometry}
\usepackage{graphicx}
\usepackage{hyperref}
\usepackage{amsmath}    % 提供方程环境
\usepackage{amssymb} 
\usepackage{physics}    % 简化向量、微分算子语法
\newcommand{\myspace}[1]{\par\vspace{#1\baselineskip}} %自定义空一行
\usepackage{fancyhdr}
\usepackage{wrapfig}
\usepackage{multirow}
\usepackage{booktabs} % 添加 booktabs 宏包
\usepackage{float}
\usepackage{subcaption}
\pagestyle{fancy}
\fancyhf{}  % 清除所有页眉页脚
\fancyfoot[C]{\thepage}  % 在页脚中央设置页码253696
\renewcommand{\headrulewidth}{0pt}  % 去掉页眉横线
\renewcommand{\footrulewidth}{0pt}  % 去掉页脚横线
% hyperref 已在上方加载一次,避免重复加载
\hypersetup{
    colorlinks=true,
    linkcolor=blue,
    filecolor=magenta,      
    urlcolor=cyan,
    pdftitle={Overleaf Example},
    pdfpagemode=FullScreen,
    }

\usepackage{titlesec}

% 重新定义section格式
\titleformat{\section}
  {\normalfont\Large\bfseries}
  {\chinese{section}、}
  {0pt}
  {}

% 给出常用字号的自定义指令
\newcommand{\chuhao}{\fontsize{42pt}{48pt}\selectfont}
\newcommand{\xiaochu}{\fontsize{36pt}{42pt}\selectfont}
\newcommand{\xiaoer}{\fontsize{18pt}{21.6pt}\selectfont}
\newcommand{\sanhao}{\fontsize{16pt}{20pt}\selectfont}
\newcommand{\xiaosan}{\fontsize{15pt}{18pt}\selectfont}
\newcommand{\sihao}{\fontsize{14pt}{18pt}\selectfont}
\newcommand{\xiaosi}{\fontsize{12pt}{16pt}\selectfont}
\newcommand{\wuhao}{\fontsize{ 10.5pt}{12.6pt}\selectfont}

% 定义中文数字转换
\titleformat{\section}
  {\normalfont\Large\bfseries}
  {\zhnum{section}、}  % 使用\zhnum而不是\chinese
  {0pt}
  {}

% 标题设置
\title{\xiaochu\textbf{物理实验报告}}
\author{}
\date{}

\begin{document}

\vspace*{36pt} % 顶部空白

% 居中填写区域
\begin{center}
    \includegraphics[width = 9cm]{figures/head.png}
    
    \xiaochu\textbf{物理实验报告}
    % 预留空白区域
    \vspace{13pt}
    \vspace{13pt}
    \vspace{13pt}
    \vspace{13pt}
    
    \sanhao\textbf{实验名称:\underline{\makebox[11cm]{ 用霍尔法测直流线圈与亥姆霍兹线圈磁场 }}} \\[10pt]
    \sanhao\textbf{实验桌号:\underline{\makebox[8cm]{  }}} \\[10pt]
    \sanhao\textbf{指导教师:\underline{\makebox[8cm]{  }}} \\[20pt]
    
    % 预留空白区域
    \myspace{7}
    
   \sihao\textbf{班级:\underline{\makebox[6cm]{cc98}}} \\[10pt]
    \sihao\textbf{姓名:\underline{\makebox[6cm]{Hydrofoil}}} \\[10pt]
    \sihao\textbf{学号:\underline{\makebox[6cm]{324010}}} \\[30pt]
    
    \sihao\textbf{实验日期: \underline{\makebox[0.8cm]{2025}}年 \underline{\makebox[0.4cm]{11}}月\underline{\makebox[0.4cm]{19}}日  星期\underline{\makebox[0.6cm]{三}}上午}
    
    \vspace{18pt}
    \sihao 浙江大学物理实验教学中心
    
\end{center}

\newpage

\section{预习报告}


    \subsection{实验综述}  

    \xiaosi (自述实验现象、实验原理和实验方法,不超过500字,5分)

    \subsubsection*{载流圆线圈与亥姆霍兹线圈的磁场:}

    \textbf{(1)载流圆线圈磁场}
    \begin{align*}
        B=\frac{\mu_{0} \cdot N_{0} \cdot I \cdot R^{2}}{2 \cdot(R^{2}+X^{2})^{\frac{3}{2}}}
    \end{align*}
    
    式中\(N_0\)为圆线圈的匝数,X为轴上某一点到圆心$o'$的距离,\(\mu_0=4\pi×10^{-7} H/m\),本实验取\(N_0=400\),\(I=0.400 A\),\(R=0.100 m\),在$O'$处\(X=0\),可计算得:\(B=1.0053×10^{-3} T\)
    
    其磁场的分布图是一条单峰的关于Y轴对称的曲线。
    
    \textbf{(2)亥姆霍兹线圈}

    定义:两个完全相同的圆线圈彼此平行且共轴,通以同方向的电流I,线圈间距等于线圈半径R时,这样的一对线圈称为亥姆霍兹线圈。其磁场分布曲线在两线圈中心连线一段出现一个平台,说明此处为匀强磁场。

    \begin{figure}[htbp]
        \centering
        % 第一张图片
        \begin{subfigure}[b]{0.45\textwidth}
            \centering
            % 替换为你的第一张文件名
            \includegraphics[width=\textwidth]{figures/载流圆线圈的磁场分布.png} 
            \caption{载流圆线圈的磁场分布}
            \label{fig:left}
        \end{subfigure}
        \hfill % 这是一个弹性空格,把两张图撑开
        % 第二张图片
        \begin{subfigure}[b]{0.45\textwidth}
            \centering
            % 替换为你的第二张文件名
            \includegraphics[width=\textwidth]{figures/亥姆霍兹线圈的磁场分布.png}
            \caption{亥姆霍兹线圈的磁场分布}
            \label{fig:right}
        \end{subfigure}
        
        \label{fig:both}
    \end{figure}

    \subsubsection*{ 利用霍尔效应测磁场的原理:}

    \textbf{(1)霍尔效应}:若电流I流过厚度为d的矩形半导体薄片,且磁场B垂直作用于该半导体,由于洛伦兹力的作用,载流子会发生横向偏转,在两横向面a、b之间产生电势差,称为霍尔电势,其方向同时垂直于I与B方向。
    
    \textbf{(2)产生原理}:I通过元件时,空穴有一定漂移速度v,其垂直于B运动时,产生洛伦兹力\(F_B=q·(v×B)\)(q为电子电荷),洛伦兹力使电荷产生横向偏转,偏转的载流子在边界积累,产生横向电场E,直到\(F_E=F_B\),即\(q·E=q·(v×B)\)。
    
    设P型样品的载流子浓度为p,宽度为w,厚度为d,通过样品的电流\(I_H=p·q·v·w·d\),
    则空穴的速度\(v=\frac{I_H}{p·q·w·d}\),代回得\(E=|v×B|=\frac{I_H·B}{p·q·w·d}\),两侧同乘w得
    \begin{align*}
        U_H=E·w=\frac{R_H·I_H·B}{d} \quad (R_H \text{称霍尔系数})
    \end{align*}
    
    应用中一般令\(k_H=\frac{R_H}{d}=\frac{1}{p·q·d}\),称为霍尔元件的灵敏度,灵敏度越大越好。
    
    由上式可知,若已知霍尔片的灵敏度\(k_H\),只要分别测出霍尔电流\(I_H\)和霍尔电势差\(U_H\),即可算出磁场B的大小,此即霍尔效应测场强的原理。

    \subsubsection*{测量载流圆线圈轴线上磁场的分布:}

    (1)正确连接 FB5 型霍尔法亥姆霍兹线圈磁场实验仪与测试架,校准微特斯拉计;

    (2)测试架左边的线圈为固定线圈,固定在 0cm 处,把右边的可动线圈移到合适位置;

    (3)使励磁电流\(I=0.400A\),以圆电流线圈中心为坐标原点,每隔 1.0cm 测一个B值,保持励磁电流不变;

    (4)记录数据,在方格纸上画出\(B-X\)曲线。

    \subsubsection*{测量亥姆霍兹线圈轴线上磁场的分布:}

    (1)移动线圈使两线圈间距\(d=R=10cm\),此时两个圆线圈中心连线的几何中心距测试平面 5cm;

    (2)将两圆电流线圈串联起来,接到磁场测试仪的输出端钮,调节电流输出,使励磁电流\(I=0.400A\),以两个圆线圈中心连线上的中点为坐标原点,每隔 1.0cm 测一个B值;

    (3)记录数据,在方格纸上画出相应的\(B-X\)曲线。

    \subsubsection*{测量载流圆线圈沿 “径向” 的磁场分布:}

    将传感器探头移动到一线圈中心,与线圈平面的夹角为 0°,径向移动探头,每移动 1.0cm 测量一个数据,按正反方向测到 6cm 为止,记录数据,作出磁场分$B-Y$曲线。
    
    \subsection{实验重点}

    \xiaosi(简述本实验的学习重点,不超过100字,3分)


    \begin{enumerate}   
        \item 了解用霍尔效应法测量磁场的原理,掌握 FB511 型霍尔法亥姆霍兹线圈磁场实验仪的使用方法;
        \item 了解载流圆线圈的径向磁场分布情况;
        \item 测量载流圆线圈和亥姆霍兹线圈的轴线上的磁场分布;
        \item 两平行线圈的间距改变为\(d=\frac{1}{2} R\)及\(d=2R\)时,测定其轴线上的磁场分布。

    \end{enumerate}  
    
    
    \subsection{实验难点}
    
    \xiaosi(简述本实验的实现难点,不超过100字,2分)

    \begin{enumerate}    
        \item 需精准校准微特斯拉计,实验中测试架位置变动后需重新调零;
        \item 霍尔传感器轴向、径向移动需精准控制,每 1cm 测一个数据,操作繁琐;
        \item 磁场易受环境干扰,探头摆放方向偏差会导致测量误差;
        \item 数据处理需绘制多组 B-X 曲线,需保证数据对称性以分析规律。
    \end{enumerate} 

    \

\section{原始数据}

    \xiaosi (将有老师签名的“自备数据记录草稿纸”的扫描或手机拍摄图粘贴在下方,20分)

    \begin{figure}[htbp]
        \centering
        % 第一张图片
        \begin{subfigure}[b]{0.49\textwidth}
            \centering
            \includegraphics[width=\textwidth]{figures/实验数据-1.jpg} 
            \label{fig:left}
        \end{subfigure}
        \hfill % 这是一个弹性空格,把两张图撑开
        % 第二张图片
        \begin{subfigure}[b]{0.49\textwidth}
            \centering
            \includegraphics[width=\textwidth]{figures/实验数据-2.jpg}
            \label{fig:right}
        \end{subfigure}
        
        \caption{实验数据}
        \label{fig:both}
    \end{figure}

\newpage
    
\section{结果与分析}

    \subsection{数据处理与结果}
    \xiaosi (列出数据表格、选择数据处理方法、给定测量或计算结果,30分)
    
    \

    整理载流圆单线圈轴线上磁场分布的数据,得出如下表格:

    % 表格第一部分
% 第一部分:左侧数据 (X < 0)
\begin{table}[h!]
    \centering
    \caption{单载流线轴线上磁场分布数据 (X < 0)}
    \setlength{\tabcolsep}{4pt}
    \renewcommand{\arraystretch}{1.3}
    \begin{tabular}{|c|c|c|c|c|c|c|c|c|c|c|c|}
    \hline
    位置 X (cm) & -10 & -9 & -8 & -7 & -6 & -5 & -4 & -3 & -2 & -1 & 0 \\ \hline
    $B_{\text{正}}$ ($\mu T$) & 312 & 371 & 442 & 511 & 600 & 671 & 779 & 854 & 900 & 951 & 971 \\ \hline
    $B_{\text{反}}$ ($\mu T$) & -403 & -462 & -518 & -602 & -675 & -779 & -842 & -916 & -975 & -1017 & -1039 \\ \hline
    \textbf{实测 $B = \frac{|B_{\text{正}}| + |B_{\text{反}}|}{2} $  (mT)} & \textbf{0.358} & \textbf{0.417} & \textbf{0.480} & \textbf{0.557} & \textbf{0.638} & \textbf{0.725} & \textbf{0.811} & \textbf{0.885} & \textbf{0.938} & \textbf{0.984} & \textbf{1.005} \\ \hline
    $B = \frac{\mu_0 N I R^2}{2(R^2 + X^2)^{3/2}}(mT)$ & 0.355 & 0.413 & 0.479 & 0.553 & 0.634 & 0.719 & 0.805 & 0.883 & 0.948 & 0.990 & 1.005 \\ \hline
    相对误差 \% & 0.85 & 0.97 & 0.21 & 0.72 & 0.63 & 0.83 & 0.75 & 0.23 & 1.05 & 0.61 & 0.00 \\ \hline
    \end{tabular}
\end{table}

% 第二部分:右侧数据 (X > 0)
\begin{table}[h!]
    \centering
    \setlength{\tabcolsep}{4pt}
    \renewcommand{\arraystretch}{1.3}
    \begin{tabular}{|c|c|c|c|c|c|c|c|c|c|c|}
    \hline
    位置 X (cm) & 1 & 2 & 3 & 4 & 5 & 6 & 7 & 8 & 9 & 10 \\ \hline
    $B_{\text{正}}$ ($\mu T$) & 959 & 925 & 860 & 780 & 699 & 619 & 533 & 465 & 389 & 338 \\ \hline
    $B_{\text{反}}$ ($\mu T$) & -1008 & -990 & -930 & -850 & -769 & -682 & -599 & -519 & -466 & -426 \\ \hline
    \textbf{实测 $B = \frac{|B_{\text{正}}| + |B_{\text{反}}|}{2} $ (mT)} & \textbf{0.984} & \textbf{0.958} & \textbf{0.895} & \textbf{0.815} & \textbf{0.734} & \textbf{0.651} & \textbf{0.566} & \textbf{0.492} & \textbf{0.428} & \textbf{0.382} \\ \hline
    $B = \frac{\mu_0 N I R^2}{2(R^2 + X^2)^{3/2}}(mT)$ & 0.990 & 0.948 & 0.883 & 0.805 & 0.719 & 0.634 & 0.553 & 0.479 & 0.413 & 0.355 \\ \hline
    相对误差 \% & 0.61 & 1.05 & 1.36 & 1.24 & 2.09 & 2.68 & 2.35 & 2.71 & 3.63 & 7.61 \\ \hline
    \end{tabular}
\end{table}

将测量值和理论值拟合成曲线,绘制于同一个坐标系上,如下图:

    \begin{figure}[H]
        \centering
        \includegraphics[width=15cm]{figures/magnetic_field_distribution.png}
        \label{figure}
    \end{figure}

    \

    整理亥姆霍兹线圈轴线上磁场分布的数据,得出如下表格:

\begin{table}[H]
    \centering
    \caption{亥姆霍兹线圈磁场分布测量数据}
    \renewcommand{\arraystretch}{1.3}
    \setlength{\tabcolsep}{3pt}
    
    % 第一部分:X = -10 到 -4
    \begin{tabular}{|c|c|c|c|c|c|c|c|}
    \hline
    位置 X (cm) & -10 & -9 & -8 & -7 & -6 & -5 & -4 \\ \hline
    $B_{\text{正}}$ ($\mu T$) & 837 & 926 & 1053 & 1163 & 1249 & 1318 & 1369 \\ \hline
    $B_{\text{反}}$ ($\mu T$) & -887 & -977 & -1101 & -1209 & -1297 & -1366 & -1415 \\ \hline
    \textbf{实测 $B = \frac{|B_{\text{正}}| + |B_{\text{反}}|}{2} $ (mT)} & \textbf{0.862} & \textbf{0.952} & \textbf{1.077} & \textbf{1.186} & \textbf{1.273} & \textbf{1.342} & \textbf{1.392} \\ \hline
    \end{tabular}

    \vspace{0.5cm}

    % 第二部分:X = -3 到 3 (中心区域)
    \begin{tabular}{|c|c|c|c|c|c|c|c|}
    \hline
    位置 X (cm) & -3 & -2 & -1 & 0 & 1 & 2 & 3 \\ \hline
    $B_{\text{正}}$ ($\mu T$) & 1397 & 1412 & 1419 & 1418 & 1412 & 1408 & 1396 \\ \hline
    $B_{\text{反}}$ ($\mu T$) & -1442 & -1460 & -1464 & -1469 & -1462 & -1458 & -1448 \\ \hline
    \textbf{实测 $B = \frac{|B_{\text{正}}| + |B_{\text{反}}|}{2} $ (mT)} & \textbf{1.420} & \textbf{1.436} & \textbf{1.442} & \textbf{1.444} & \textbf{1.437} & \textbf{1.433} & \textbf{1.422} \\ \hline
    \end{tabular}

    \vspace{0.5cm}

    % 第三部分:X = 4 到 10
    \begin{tabular}{|c|c|c|c|c|c|c|c|}
    \hline
    位置 X (cm) & 4 & 5 & 6 & 7 & 8 & 9 & 10 \\ \hline
    $B_{\text{正}}$ ($\mu T$) & 1381 & 1337 & 1275 & 1192 & 1089 & 983 & 872 \\ \hline
    $B_{\text{反}}$ ($\mu T$) & -1432 & -1384 & -1326 & -1241 & -1144 & -1044 & -937 \\ \hline
    \textbf{实测 $B = \frac{|B_{\text{正}}| + |B_{\text{反}}|}{2} $ (mT)} & \textbf{1.407} & \textbf{1.361} & \textbf{1.301} & \textbf{1.217} & \textbf{1.117} & \textbf{1.014} & \textbf{0.905} \\ \hline
    \end{tabular}

\end{table}

    将测量值和理论值拟合成曲线,绘制于同一个坐标系上,如下图:

    \begin{figure}[H]
        \centering
        \includegraphics[width = \textwidth]{figures/helmholtz_fit.png}
        \label{figure}
    \end{figure}

    整理载流圆线圈中心平面内径向磁场分布的数据,得出如下表格:

\begin{table}[H]
    \centering
    \caption{磁场径向分布测量数据 (Y 轴方向)}
    \renewcommand{\arraystretch}{1.4}
    \setlength{\tabcolsep}{4pt}
    
    % 第一部分:Y = -5.0 到 2.0
    \begin{tabular}{|c|c|c|c|c|c|c|c|c|c|c|c|c|}
    \hline
    径向距离 Y ($10^{-2}$ m) & -5.0 & -4.0 & -3.0 & -2.0 & -1.0 & 0.0 & 1.0 & 2.0 & 3.0 & 4.0 & 5.0\\ \hline
    $B = \frac{|B_{\text{正}}| + |B_{\text{反}}|}{2} $ (mT) & 1.207 & 1.117 & 1.060 & 1.022 & 1.006 & 1.002 & 1.014 & 1.041 & 1.095 & 1.167 & 1.291\\ \hline
    \end{tabular}
\end{table}

    将测量值和理论值拟合成曲线,绘制于同一个坐标系上,如下图:

    \begin{figure}[H]
        \centering
        \includegraphics[width = \textwidth]{figures/radial_distribution_fit.png}
        \label{figure}
    \end{figure}

    \textbf{实验结论:}
    
    (1)载流圆单线圈的轴向磁场分布关于线圈中心对称,且中心处磁感应强度最大,沿轴向距中心越远,磁感应强度越小;
    
    (2)亥姆霍兹线圈轴线上磁场的分布关于中心对称,且当两线圈间距\(d=R\)时,两线圈合磁场在中心轴线上(两线圈圆心连线)附近较大范围内是均匀的,图像上呈现出一段平台期;
    
    (3)载流圆单线圈中心平面内径向磁场分布关于中心对称,且中心处磁感应强度最小,沿径向距中心越远,磁感应强度越大。

    \subsection{误差分析}
    \xiaosi (运用测量误差、相对误差、不确定度等分析实验结果,20分)
    
    \begin{enumerate}
        \item 由于磁场易受到环境影响,测出的磁感应强度与计算得到的理论值存在一定的误差;
        \item 测量各个点的磁感应强度时,探头的摆放方式与方向存在少许偏差,故每一点的测量都会因方向不同而产生不同程度的误差;
        \item 调零时会有 ±2μT 的示数浮动,测量时的值也会存在浮动,不够稳定;
        \item 调节霍尔元件位置时,可能存在视觉误差,且仪器整体结构存在松动情况,会造成一定误差。
    \end{enumerate}   
    
    \subsection{实验探讨}
    \xiaosi (对实验内容、现象和过程的小结,不超过100字,10分)

    \myspace{1}

    本次实验操作相对较简单,但需测量较多数据,极大地锻炼了我的观察能力、数据处理能力,提高了我的实验综合素养,可谓收获颇丰!

    \section{思考题}

    \xiaosi (解答教材或讲义或老师布置的思考题,10分)

    \subsubsection*{为什么在测量直流磁场时,必须考虑地球磁场对被测磁场的影响?}

    \textbf{答:}地球磁场是一个恒定的直流磁场,磁场与磁场之间会产生相互干扰,且通电导线产生的磁场较小,故地球磁场不可忽略。

    \subsubsection*{载流圆线圈轴线上磁场的分布规律如何?}

    \textbf{答:}距离中心越远,磁场强度越小,且其关于中心对称,在中心处的磁场强度最大。

    \subsubsection*{亥姆霍兹线圈是怎样组成的?其基本条件有哪些?它的磁场分布特点又怎样?改变两圆线圈间距后,线圈轴线上的磁场分布情况如何?}

    \textbf{答:}亥姆霍兹线圈为两个相同的线圈彼此平行且共轴,通以同方向电流I而构成;基本条件是线圈完全相同、平行共轴、通同方向电流,且间距等于线圈半径R;磁场分布特点是轴上两线圈圆心连线附近较大范围内为匀强磁场;若减小间距,中心轴线上的均匀磁场区域会缩小,中心点的磁场强度增加;若增大间距,中心轴线上的均匀磁场区域会扩大,中心点的磁场强度减少(若间距过大,两线圈的磁场效应变得相对独立,中心点附近磁场均匀性会变差)。
    
    \subsubsection*{霍尔元件放入磁场时, 不同方向上特斯拉计指示值不同,哪个方向最大?}

    \textbf{答:}探测器探头与磁场方向垂直时示数最大。

    \subsubsection*{试分析载流圆线圈磁场分布的理论值与实验值的误差产生的原因?}

    \textbf{答:}霍尔元件可能因使用时间过长导致测量数据不准;环境中可能存在其它电磁场,进而产生干扰。


\newpage

\sihao\textbf{注意事项:}
\xiaosi
\begin{enumerate}
    \item  用 PDF 格式上传“实验报告”,文件名:学生姓名+学号+实验名称+周次。
    \item  “实验报告”必须递交在“学在浙大”的本课程的对应实验项目的“作业”模块内。
    \item  “实验报告”成绩必须在“浙江大学物理实验教学中心网站”-“选课系统”内查询。
    \item 教学评价必须在“浙江大学物理实验教学中心网站”-“选课系统”内进行,学生必须进行教学评价,才能看到实验报告成绩,教学评价必须在本次实验结束后 3 天内进行。
\end{enumerate}

\vspace{1cm}
\xiaosi\centering \textbf{浙江大学物理实验教学中心制}

\end{document}