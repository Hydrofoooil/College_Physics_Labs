\documentclass[10pt,a4paper]{ctexart}
\usepackage[margin=2cm]{geometry}
\usepackage{graphicx}
\usepackage{hyperref}
\usepackage{amsmath}    % 提供方程环境
\usepackage{amssymb}    % 提供数学符号, 如 \therefore
\usepackage{physics}    % 简化向量、微分算子语法
\newcommand{\myspace}[1]{\par\vspace{#1\baselineskip}} %自定义空一行
\usepackage{fancyhdr}
\usepackage{wrapfig}
\usepackage{booktabs} % 添加 booktabs 宏包
\usepackage{float} % 引入float宏包以使用[H]选项
\pagestyle{fancy}
\fancyhf{}  % 清除所有页眉页脚
\fancyfoot[C]{\thepage}  % 在页脚中央设置页码
\renewcommand{\headrulewidth}{0pt}  % 去掉页眉横线
\renewcommand{\footrulewidth}{0pt}  % 去掉页脚横线

\usepackage{subcaption}
\usepackage{hyperref}
\hypersetup{
    colorlinks=true,
    linkcolor=blue,
    filecolor=magenta,      
    urlcolor=cyan,
    pdftitle={Overleaf Example},
    pdfpagemode=FullScreen,
    }

\usepackage{titlesec}

% 重新定义section格式
\titleformat{\section}
  {\normalfont\Large\bfseries}
  {\chinese{section}、}
  {0pt}
  {}

% 给出常用字号的自定义指令
\newcommand{\chuhao}{\fontsize{42pt}{48pt}\selectfont}
\newcommand{\xiaochu}{\fontsize{36pt}{42pt}\selectfont}
\newcommand{\xiaoer}{\fontsize{18pt}{21.6pt}\selectfont}
\newcommand{\sanhao}{\fontsize{16pt}{20pt}\selectfont}
\newcommand{\xiaosan}{\fontsize{15pt}{18pt}\selectfont}
\newcommand{\sihao}{\fontsize{14pt}{18pt}\selectfont}
\newcommand{\xiaosi}{\fontsize{12pt}{16pt}\selectfont}
\newcommand{\wuhao}{\fontsize{10.5pt}{12.6pt}\selectfont}

% 定义中文数字转换
\titleformat{\section}
  {\normalfont\Large\bfseries}
  {\zhnum{section}、}  % 使用\zhnum而不是\chinese
  {0pt}
  {}

% 标题设置
\title{\xiaochu\textbf{物理实验预习报告}}
\author{}
\date{}

\begin{document}

\vspace*{36pt} % 顶部空白

% 居中填写区域
\begin{center}
    \includegraphics[width = 9cm]{figures/head.png}
    
    \xiaochu\textbf{物理实验预习报告}
    % 预留空白区域
    \vspace{13pt}
    \vspace{13pt}
    \vspace{13pt}
    \vspace{13pt}
    
\sanhao\textbf{实验名称:\underline{\makebox[8cm]{ 万用表的设计 }}} \\[10pt]
    \sanhao\textbf{实验桌号:\underline{\makebox[8cm]{  }}} \\[10pt]
    \sanhao\textbf{指导教师:\underline{\makebox[8cm]{  }}} \\[20pt]
    
    % 预留空白区域
    \myspace{7}
    
   \sihao\textbf{班级:\underline{\makebox[6cm]{cc98}}} \\[10pt]
    \sihao\textbf{姓名:\underline{\makebox[6cm]{Hydrofoil}}} \\[10pt]
    \sihao\textbf{学号:\underline{\makebox[6cm]{324010}}} \\[30pt]
    
   \sihao\textbf{实验日期: \underline{\makebox[0.8cm] 2025 }年 \underline{\makebox[0.4cm]11}月\underline{\makebox[0.4cm]26}日  星期\underline{\makebox[0.6cm]三}上午}
    
    \vspace{18pt}
    \sihao 浙江大学物理实验教学中心
    
\end{center}

\newpage


    \section{实验综述}
    
    \xiaosi (自述实验现象、实验原理和实验方法,不超过500字,5分)
    
    \myspace{0}

    
    \subsubsection*{改装多量程电流表原理:}

    
    要将磁电式电流计改装成量程为 I 的电流表,只需在电表表头两端并联一个分流电阻,其\(R_{s}=\frac{R_{g} I_{g}}{I - I_{g}}\),并联不同的分流电阻可构成不同量程的电流表。
    
    如下图,\(\begin{cases}(R_{1}+R_{2})(I_{2}-I_{g})=R_{g}I_{g} \\ R_{1}(I_{1}-I_{g})=(R_{2}+R_{g})I_{g} \end{cases}\) 
    可得到\(R_{1}\)、\(R_{2}\)的值,
    最后用标准电流表对改装的电流表进行校正。
    \begin{figure}[htbp]
        \centering
        \begin{subfigure}[t]{0.45\textwidth}
            \centering
            \includegraphics[width=0.9\textwidth]{figures/改装电流表1.png}
            \label{fig:current_meter1}
        \end{subfigure}
        \begin{subfigure}[t]{0.45\textwidth}
            \centering
            \includegraphics[width=0.8\textwidth]{figures/改装电流表2.png}
            \label{fig:current_meter2}
        \end{subfigure}
        \caption{改装电路表电路图}
        \label{fig:图1}
    \end{figure}

    \subsubsection*{改装多量程电压表原理:}

    要将磁电式电流计改装成量程为 U 的电压表,则电流计需串联一个分压电阻,
    其\(R_{x}=\frac{U}{I_{g}}-R_{g}\),串联不同的分压电阻可得到不同量程的电压表。
    
    如图,有:\( \begin{cases} R_{3}I_{g}=U_{1}-R_{g}I_{g} \\ R_{4}I_{g}=U_{2}-R_{g}I_{g} \end{cases} \)(若用 1 中改装的电流表代替表头,则记为 \(R_{g}'\)和\(I_{g}'\) ),
    
    
    最后用标准伏特表对改装的电压表进行校正。

    \begin{figure}[htbp]
        \centering
        \begin{subfigure}[b]{0.45\textwidth}
            \centering
            \includegraphics[width=1.1\textwidth]{figures/改装电压表1.png}
            \label{fig:current_meter1}
        \end{subfigure}
        \begin{subfigure}[t]{0.45\textwidth}
            \centering
            \includegraphics[width=0.7\textwidth]{figures/改装电压表2.png}
            \label{fig:current_meter2}
        \end{subfigure}
        \caption{改装电路表电路图}
        \label{fig:图1}
    \end{figure}

    \subsubsection*{改装欧姆表原理:}
    
    \begin{wrapfigure}{r}{0.4\textwidth}
        \vspace{-5pt} % 向上移动图片,减少图片上方的空白
        \centering
        \includegraphics[width=0.5\textwidth]{figures/改装欧姆表.png}
        \caption{欧姆表校准电路图}
        \label{fig:ohm_meter_cal}
    \end{wrapfigure}
    
    短接 a、b 两端,调节电阻 R 使电流计满刻度,此时总电阻为\(R_{0}=R_{g}+R'\),
    则当接入电阻\(R_{x}\)后,回路电流\(I_{x}=\frac{\varepsilon}{R_{0}+R_{x}}\)。
    当\(R_{x}=R_{g}+R'\)时,\(I_{x}=\frac{I_{g}}{2}\),此时电表指针指向刻度线中点,这时的电阻\(R_{x}\)称为欧姆表的中值电阻。
    
    由于\(I_{x}\)与\(R_{x}\)呈非线性关系,所以欧姆表刻度为非均匀刻度。
    由于作为电源的电池也非恒定,所以欧姆表还需作零欧姆调整,通过可调电位器\(R_{6}\)实现。
    
    若要扩大欧姆表量程,可采用两种方法:一是电流计两端并联不同的分流电阻,二是可提高电源电压。

    \subsubsection*{设计多量程电流表并校准:}

    实验室提供 1mA 量程电流计,内阻已标注。若要重新测量电流计内阻改装电流表实验值,可采用替代法测量,即:先将电流计与标准电流表同时串接在测量回路中,调整回路电流到合适大小 I,然后用电阻箱将电流计换下,改变阻值使 I 不变,此时电阻箱的阻值即为电流计的内阻。

    \begin{enumerate}
        \item 用满偏法和替代法测量表头的\(I_{g}\)、\(R_{g}\);
        \item 计算多量程电流表的分流电阻;
        \item 校准与记录,计算\(\Delta I\),确定\(k=\frac{\Delta I_{max}}{\text{量程}}×100\%\)。
    \end{enumerate}

    \subsubsection*{设计多量程电压表并校准:}

    \begin{enumerate}
        \item 计算多量程电压表的分压电阻;
        \item 校准与记录,记录\(\Delta V\)与 U,绘制校准曲线,确定\(k=\frac{\Delta V_{max}}{\text{量程}}×100\%\)。
    \end{enumerate}

    \subsubsection*{设计欧姆表并制作欧姆档刻度曲线:}

    \(I_{x}\)与\(R_{x}\)成非线性关系,记录多组\(R_{x}\)及\(I_{x}\)的数据,用此方法,可在电流计面板上刻刻度,以显示不同的阻值\(R_{x}\),且刻度为非均匀刻度。另外,实际作为电源的电池非恒定,故还需作零欧姆调整。


    \section{实验重点}
    
    \xiaosi(简述本实验的学习重点,不超过100字,3分)

    \begin{enumerate}
        \item 了解指针式万用表测量电流、电压以及电阻的基本原理;
        \item 掌握多量程电流表、电压表和万用表的设计方法。
    \end{enumerate}
    
    \section{实验难点}
    
    \xiaosi(简述本实验的实现难点,不超过100字,2分)

    \begin{enumerate}
        \item 计算分流、分压电阻需精准,数值偏差会影响电表量程准确性;
        \item 校准电表时,要控制电流 / 电压稳定,绘制校准曲线需多次精准记录数据;
        \item 欧姆表刻度非线性,制作刻度曲线需处理多组数据,且电池需频繁零欧姆调整;
        \item 测量电流计内阻时,替代法操作需保证回路电流不变,难度较高。
    \end{enumerate}



\newpage

\sihao\textbf{注意事项:}
\xiaosi
\begin{enumerate}
    \item  用 PDF 格式上传“实验报告”,文件名:学生姓名+学号+实验名称+周次。
    \item  “实验报告”必须递交在“学在浙大”的本课程的对应实验项目的“作业”模块内。
    \item  “实验报告”成绩必须在“浙江大学物理实验教学中心网站”-“选课系统”内查询。
    \item 教学评价必须在“浙江大学物理实验教学中心网站”-“选课系统”内进行,学生必须进行教学评价,才能看到实验报告成绩,教学评价必须在本次实验结束后 3 天内进行。
\end{enumerate}

\vspace{1cm}
\xiaosi\centering \textbf{浙江大学物理实验教学中心制}

\end{document}