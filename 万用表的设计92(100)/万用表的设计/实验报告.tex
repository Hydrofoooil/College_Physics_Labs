\documentclass[10pt,a4paper]{ctexart}
\usepackage[margin=2cm]{geometry}
\usepackage{graphicx}
\usepackage{hyperref}
\usepackage{amsmath}    % 提供方程环境
\usepackage{amssymb}    % 提供数学符号, 如 \therefore
\usepackage{physics}    % 简化向量、微分算子语法
\newcommand{\myspace}[1]{\par\vspace{#1\baselineskip}} %自定义空一行
\usepackage{fancyhdr}
\usepackage{wrapfig}
\usepackage{booktabs} % 添加 booktabs 宏包
\usepackage{float} % 引入float宏包以使用[H]选项
\pagestyle{fancy}
\fancyhf{}  % 清除所有页眉页脚
\fancyfoot[C]{\thepage}  % 在页脚中央设置页码
\renewcommand{\headrulewidth}{0pt}  % 去掉页眉横线
\renewcommand{\footrulewidth}{0pt}  % 去掉页脚横线

\usepackage{subcaption}
\usepackage{hyperref}
\hypersetup{
    colorlinks=true,
    linkcolor=blue,
    filecolor=magenta,      
    urlcolor=cyan,
    pdftitle={Overleaf Example},
    pdfpagemode=FullScreen,
    }

\usepackage{titlesec}

% 重新定义section格式
\titleformat{\section}
  {\normalfont\Large\bfseries}
  {\chinese{section}、}
  {0pt}
  {}

% 给出常用字号的自定义指令
\newcommand{\chuhao}{\fontsize{42pt}{48pt}\selectfont}
\newcommand{\xiaochu}{\fontsize{36pt}{42pt}\selectfont}
\newcommand{\xiaoer}{\fontsize{18pt}{21.6pt}\selectfont}
\newcommand{\sanhao}{\fontsize{16pt}{20pt}\selectfont}
\newcommand{\xiaosan}{\fontsize{15pt}{18pt}\selectfont}
\newcommand{\sihao}{\fontsize{14pt}{18pt}\selectfont}
\newcommand{\xiaosi}{\fontsize{12pt}{16pt}\selectfont}
\newcommand{\wuhao}{\fontsize{10.5pt}{12.6pt}\selectfont}

% 定义中文数字转换
\titleformat{\section}
  {\normalfont\Large\bfseries}
  {\zhnum{section}、}  % 使用\zhnum而不是\chinese
  {0pt}
  {}

% 标题设置
\title{\xiaochu\textbf{物理实验报告}}
\author{}
\date{}

\begin{document}

\vspace*{36pt} % 顶部空白

% 居中填写区域
\begin{center}
    \includegraphics[width = 9cm]{figures/head.png}
    
    \xiaochu\textbf{物理实验报告}
    % 预留空白区域
    \vspace{13pt}
    \vspace{13pt}
    \vspace{13pt}
    \vspace{13pt}
    
    \sanhao\textbf{实验名称:\underline{\makebox[8cm]{ 万用表的设计 }}} \\[10pt]
    \sanhao\textbf{实验桌号:\underline{\makebox[8cm]{  }}} \\[10pt]
    \sanhao\textbf{指导教师:\underline{\makebox[8cm]{  }}} \\[20pt]
    
    % 预留空白区域
    \myspace{7}
    
   \sihao\textbf{班级:\underline{\makebox[6cm]{cc98}}} \\[10pt]
    \sihao\textbf{姓名:\underline{\makebox[6cm]{Hydrofoil}}} \\[10pt]
    \sihao\textbf{学号:\underline{\makebox[6cm]{324010}}} \\[30pt]
    
   \sihao\textbf{实验日期: \underline{\makebox[0.8cm] 2025 }年 \underline{\makebox[0.4cm]11}月\underline{\makebox[0.4cm]26}日  星期\underline{\makebox[0.6cm]三}上午}
    
    \vspace{18pt}
    \sihao 浙江大学物理实验教学中心
    
\end{center}

\newpage

\section{预习报告}

    \subsection{实验综述}
    
    \xiaosi (自述实验现象、实验原理和实验方法,不超过500字,5分)
    
    \myspace{0}
    
    \subsubsection*{改装多量程电流表原理:}

    
    要将磁电式电流计改装成量程为 I 的电流表,只需在电表表头两端并联一个分流电阻,其\(R_{s}=\frac{R_{g} I_{g}}{I - I_{g}}\),并联不同的分流电阻可构成不同量程的电流表。
    
    如下图,\(\begin{cases}(R_{1}+R_{2})(I_{2}-I_{g})=R_{g}I_{g} \\ R_{1}(I_{1}-I_{g})=(R_{2}+R_{g})I_{g} \end{cases}\) 
    可得到\(R_{1}\)、\(R_{2}\)的值,
    最后用标准电流表对改装的电流表进行校正。
    \begin{figure}[htbp]
        \centering
        \begin{subfigure}[t]{0.45\textwidth}
            \centering
            \includegraphics[width=0.9\textwidth]{figures/改装电流表1.png}
            \label{fig:current_meter1}
        \end{subfigure}
        \begin{subfigure}[t]{0.45\textwidth}
            \centering
            \includegraphics[width=0.8\textwidth]{figures/改装电流表2.png}
            \label{fig:current_meter2}
        \end{subfigure}
        \caption{改装电路表电路图}
        \label{fig:图1}
    \end{figure}

    \subsubsection*{改装多量程电压表原理:}

    要将磁电式电流计改装成量程为 U 的电压表,则电流计需串联一个分压电阻,
    其\(R_{x}=\frac{U}{I_{g}}-R_{g}\),串联不同的分压电阻可得到不同量程的电压表。
    
    如图,有:\( \begin{cases} R_{3}I_{g}=U_{1}-R_{g}I_{g} \\ R_{4}I_{g}=U_{2}-R_{g}I_{g} \end{cases} \)(若用 1 中改装的电流表代替表头,则记为 \(R_{g}'\)和\(I_{g}'\) ),
    
    
    最后用标准伏特表对改装的电压表进行校正。

    \begin{figure}[htbp]
        \centering
        \begin{subfigure}[b]{0.45\textwidth}
            \centering
            \includegraphics[width=1.1\textwidth]{figures/改装电压表1.png}
            \label{fig:current_meter1}
        \end{subfigure}
        \begin{subfigure}[t]{0.45\textwidth}
            \centering
            \includegraphics[width=0.7\textwidth]{figures/改装电压表2.png}
            \label{fig:current_meter2}
        \end{subfigure}
        \caption{改装电路表电路图}
        \label{fig:图1}
    \end{figure}

    \subsubsection*{改装欧姆表原理:}
    
    \begin{wrapfigure}{r}{0.4\textwidth}
        \vspace{-5pt} % 向上移动图片,减少图片上方的空白
        \centering
        \includegraphics[width=0.5\textwidth]{figures/改装欧姆表.png}
        \caption{欧姆表校准电路图}
        \label{fig:ohm_meter_cal}
    \end{wrapfigure}
    
    短接 a、b 两端,调节电阻 R 使电流计满刻度,此时总电阻为\(R_{0}=R_{g}+R'\),
    则当接入电阻\(R_{x}\)后,回路电流\(I_{x}=\frac{\varepsilon}{R_{0}+R_{x}}\)。
    当\(R_{x}=R_{g}+R'\)时,\(I_{x}=\frac{I_{g}}{2}\),此时电表指针指向刻度线中点,这时的电阻\(R_{x}\)称为欧姆表的中值电阻。
    
    由于\(I_{x}\)与\(R_{x}\)呈非线性关系,所以欧姆表刻度为非均匀刻度。
    由于作为电源的电池也非恒定,所以欧姆表还需作零欧姆调整,通过可调电位器\(R_{6}\)实现。
    
    若要扩大欧姆表量程,可采用两种方法:一是电流计两端并联不同的分流电阻,二是可提高电源电压。

    \subsubsection*{设计多量程电流表并校准:}

    实验室提供 1mA 量程电流计,内阻已标注。若要重新测量电流计内阻改装电流表实验值,可采用替代法测量,即:先将电流计与标准电流表同时串接在测量回路中,调整回路电流到合适大小 I,然后用电阻箱将电流计换下,改变阻值使 I 不变,此时电阻箱的阻值即为电流计的内阻。

    \begin{enumerate}
        \item 用满偏法和替代法测量表头的\(I_{g}\)、\(R_{g}\);
        \item 计算多量程电流表的分流电阻;
        \item 校准与记录,计算\(\Delta I\),确定\(k=\frac{\Delta I_{max}}{\text{量程}}×100\%\)。
    \end{enumerate}

    \subsubsection*{设计多量程电压表并校准:}

    \begin{enumerate}
        \item 计算多量程电压表的分压电阻;
        \item 校准与记录,记录\(\Delta V\)与 U,绘制校准曲线,确定\(k=\frac{\Delta V_{max}}{\text{量程}}×100\%\)。
    \end{enumerate}

    \subsubsection*{设计欧姆表并制作欧姆档刻度曲线:}

    \(I_{x}\)与\(R_{x}\)成非线性关系,记录多组\(R_{x}\)及\(I_{x}\)的数据,用此方法,可在电流计面板上刻刻度,以显示不同的阻值\(R_{x}\),且刻度为非均匀刻度。另外,实际作为电源的电池非恒定,故还需作零欧姆调整。

    \subsection{实验重点}
    
    \xiaosi(简述本实验的学习重点,不超过100字,3分)

    \begin{enumerate}
        \item 了解指针式万用表测量电流、电压以及电阻的基本原理;
        \item 掌握多量程电流表、电压表和万用表的设计方法。
    \end{enumerate}
    
    \subsection{实验难点}
    
    \xiaosi(简述本实验的实现难点,不超过100字,2分)

    \begin{enumerate}
        \item 计算分流、分压电阻需精准,数值偏差会影响电表量程准确性;
        \item 校准电表时,要控制电流 / 电压稳定,绘制校准曲线需多次精准记录数据;
        \item 欧姆表刻度非线性,制作刻度曲线需处理多组数据,且电池需频繁零欧姆调整;
        \item 测量电流计内阻时,替代法操作需保证回路电流不变,难度较高。
    \end{enumerate}


\section{原始数据}

    \xiaosi (将有老师签名的“自备数据记录草稿纸”的扫描或手机拍摄图粘贴在下方,20分)

    \begin{figure}[H]
        \centering
        \includegraphics[width=13.5cm]{figures/实验数据.jpg}
        \caption{实验数据}
        \label{fig:图2}
    \end{figure}

    
\section{结果与分析}

    \subsection{数据处理与结果}
    \xiaosi (列出数据表格、选择数据处理方法、给定测量或计算结果,30分)

    \subsubsection*{改装多量程电流表}

    由半偏法测得$R_g = 216\Omega$, 且$I_g=1.00mA$
    
    则由\(\begin{cases}(R_{1}+R_{2})(5-I_{g})=R_{g}I_{g} \\ R_{1}(10-I_{g})=(R_{2}+R_{g})I_{g}\end{cases}\) 
    可得到\(R_{1}=R_{2}=\frac{1}{8}R_g=27\Omega\)

    对改装后电流表进行校准可得下表数据:

    % 这是一个完整的、可浮动的表格环境
    \begin{table}[htbp]
        \centering % 让表格居中显示
        \label{tab:resistance_data} % 表格的标签,用于交叉引用
        \begin{tabular}{*{6}{c}} % 定义表格有3列,每一列都居中(c)
            \toprule % 画出顶部的粗横线 (来自 booktabs)
            $I_G/mA$ & 1.00 & 3.00 & 3.00 & 4.00 & 4.50 \\
            \midrule % 画出中间的分割线 (来自 booktabs)
            $I_B/mA$ & 1.00 & 2.00 & 2.95 & 4.05 & 4.60 \\
            $\Delta I/mA$ & 0.00 & 0.00 & -0.05 & 0.05 & 0.10 \\
            $R_2 / \Omega $ & 27  & 27  & 27 &27 &27 \\
            \bottomrule % 画出底部的粗横线 (来自 booktabs)
        \end{tabular}
    \end{table}

由对于校准结果:
\begin{center}
$|\Delta I_{max}|=0.10mA $\\
$\therefore K_1 = \frac{|\Delta I_{max}|}{\text{A量程}}\times 100\% = 2.00\%$
\end{center}

绘制$I_G - I_B$曲线如下:

\begin{figure}[htbp]
    \centering    
        \includegraphics[width=0.9\textwidth]{figures/linear_fit_chart1.jpg}
    \label{fig:图1}
\end{figure}



\subsubsection*{改装多量程电压表}

    已知\(I_{g}'=5mA\),\(R_{g}'=\frac{R_{g}(R_{1}+R_{2})}{R_{4}+R_{1}+R_{2}} \approx 47.98\Omega\)

    则由\( \begin{cases} R_{3}I_{g}=U_{1}-R_{g}I_{g} \\ R_{4}I_{g}=U_{2}-R_{g}I_{g} \end{cases} \),得到$\begin{cases} R_3 = 950\Omega \\ R_4 = 1000\Omega \end{cases}$

    对改装后电压表进行校准可得下表数据:

    % 这是一个完整的、可浮动的表格环境
    \begin{table}[htbp]
        \centering % 让表格居中显示
        \label{tab:resistance_data} % 表格的标签,用于交叉引用
        \begin{tabular}{*{6}{c}} % 定义表格有3列,每一列都居中(c)
            \toprule % 画出顶部的粗横线 (来自 booktabs)
            $V_G/V$ & 1.00 & 3.00 & 3.00 & 4.00 & 4.50 \\
            \midrule % 画出中间的分割线 (来自 booktabs)
            $V_B/V$ & 1.01 & 2.01 & 3.04 & 4.00 & 4.53 \\
            $\Delta V/V$ & 0.01 & 0.01 & 0.04 & 0.00 & 0.03 \\
            $R_3 / \Omega $ & 950  & 950  & 950 & 950 & 950 \\
            \bottomrule % 画出底部的粗横线 (来自 booktabs)
        \end{tabular}
    \end{table}

由对于校准结果:
\begin{center}
$|\Delta V_{max}|=0.104V $\\
$\therefore K_2 = \frac{|\Delta V_{max}|}{\text{V量程}}\times 100\% = 0.80\%$
\end{center}

绘制$V_G - V_B$曲线如下:

\begin{figure}[htbp]
    \centering    
        \includegraphics[width=0.9\textwidth]{figures/linear_fit_chart2.jpg}
    \label{fig:图1}
\end{figure}

    \subsubsection*{设计欧姆表}

    当短接欧姆表两输入端,调 \(R_{6}\) 使电流计满偏时, 计算得\(I_{0}=\frac{\varepsilon}{R_{g}'+R'}\) 即$ R_g' + R' =\frac{\varepsilon}{I_0} = \frac{1.5V}{5mA}=300\Omega$ 。当\(R_x=R_g'+R'\)时 ,电流计半偏 \(I_{x}=\frac{x_{0}}{2}\) 即中值电阻 \(r=R_g'+R'=300\Omega\) ,与实际情况相符。 \((R'=R_{5}+R_{6}\) )

    % 这是一个完整的、可浮动的表格环境
    \begin{table}[htbp]
        \centering % 让表格居中显示
        \label{tab:resistance_data} % 表格的标签,用于交叉引用
        \begin{tabular}{*{12}{c}} % 定义表格有3列,每一列都居中(c)
            \toprule % 画出顶部的粗横线 (来自 booktabs)
            $I_x/mA$ & 5.00 & 4.50 & 4.00 & 3.50 & 3.00 & 2.50 & 2.00 & 1.50 & 1.00 & 0.50 & 0.00\\
            \midrule % 画出中间的分割线 (来自 booktabs)
            $R_x / \Omega $ & 0  & 31.9  & 82.9 & 144.8 & 223.1 & 327.8 & 478.6 & 735.0 & 1257.2 & 2810.1& $ \infty $\\
            \bottomrule % 画出底部的粗横线 (来自 booktabs)
        \end{tabular}
    \end{table}

    绘制$I_x - R_x$曲线如下:

\begin{figure}[htbp]
    \centering    
        \includegraphics[width=0.9\textwidth]{figures/nonlinear_fit_chart.jpg}
    \label{fig:图1}
\end{figure}

    由此,可在电流计面板上刻上刻度,以显示不同的阻值$R_x$ ,不过由于$I_x$与 $R_x$ 呈非线性关系,故欧姆表刻度为非均匀刻度。
    \subsection{误差分析}
    \xiaosi (运用测量误差、相对误差、不确定度等分析实验结果,20分)

    \myspace{1}

    \begin{enumerate}
        \item 接入的导线存在电阻,有一定损耗;
        \item 电元件可能随使用时间增长而老化,导致电流计的标称阻止与实际阻值存在偏差;
        \item 仪器的精度不够高,以及电流计游丝弹性偏大或偏小,导致电流计游丝浮动或指示不精确;
        \item 读数时视线不能始终保持与表盘刻度垂直。读数存在偶然误差;
        \item 标准电流表、电压装的机械零件位移,导致其零点漂移,校正不准确。
    \end{enumerate}

    \subsection{实验探讨}
    \xiaosi (对实验内容、现象和过程的小结,不超过100字,10分)

    \myspace{1}

    通过次实验,我明白了指针式方用表测量电流、电压和电阻的基本原理,掌握了多量程电流表、电压表和欧姆表的设计方法。实验中实际电路的设计、导线与器件的连接和电阻的选取调整等极大锻炼了我的动手操作能力,进步巩固了我电学相关知识。同时,调整校准电表、滑动变阻器需要极高的操作精度与时刻不停的观察,稍有不慎便会造成较大误差,而这也锻炼了我的耐心与观察能力。   
    
    \section{思考题}

    \xiaosi (解答教材或讲义或老师布置的思考题,10分)

    \subsubsection*{为什么不能用万用表欧姆档测量电源内阻?}
    
    \textbf{答:} 万用表欧姆档自带电源,本质上是由测得的不同电流值确定不同的电阻。在测量电源电阻时外加电源的电压会与表内电源的电压发生干扰,甚至可能直接损坏电表,故无法测出其真实阻值。
    
    \subsubsection*{为什么不能用欧姆表测量另一表头内阻?}
    
    \textbf{答:} 因为欧姆表的电阻刻度不均,测量电阻时只能估读,无法获得精确值(尤其在测量小电阻时误差极大)。而表头用于改装电表,对其阻值测量值度要求较高,故不能用欧姆表测另一表头内阻。

    \subsubsection*{为什么$I_x$与 $R_x$ 呈非线性关系?}

    \textbf{答:} 因为$I_x$与 $R_x$ 满足关系式\(I_{x}=\frac{\varepsilon}{R_{x}'+R_0'}\)。其中$R_0$为电流计满偏时的总阻值 $R_0=\frac{\varepsilon}{I_0}$(包括改装电压表等效内阻、限流电阻、欧姆调零电阻、电源 内阻等所有回路中的电阻),由$R_x=\frac{\varepsilon}{I_0}-R_0$可知:$\frac{1}{I_x}$与$R_x$成线性关系,而$I_x$与$R_x$为非线性关系。
  

\newpage

\sihao\textbf{注意事项:}
\xiaosi
\begin{enumerate}
    \item  用 PDF 格式上传“实验报告”,文件名:学生姓名+学号+实验名称+周次。
    \item  “实验报告”必须递交在“学在浙大”的本课程的对应实验项目的“作业”模块内。
    \item  “实验报告”成绩必须在“浙江大学物理实验教学中心网站”-“选课系统”内查询。
    \item 教学评价必须在“浙江大学物理实验教学中心网站”-“选课系统”内进行,学生必须进行教学评价,才能看到实验报告成绩,教学评价必须在本次实验结束后 3 天内进行。
\end{enumerate}

\vspace{1cm}
\xiaosi\centering \textbf{浙江大学物理实验教学中心制}

\end{document}