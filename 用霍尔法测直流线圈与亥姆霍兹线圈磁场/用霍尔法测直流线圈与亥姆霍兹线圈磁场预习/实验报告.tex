\documentclass[10pt,a4paper]{ctexart}
\usepackage[margin=2cm]{geometry}
\usepackage{graphicx}
\usepackage{hyperref}
\usepackage{amsmath}    % 提供方程环境
\usepackage{physics}    % 简化向量、微分算子语法
\newcommand{\myspace}[1]{\par\vspace{#1\baselineskip}} %自定义空一行
\usepackage{fancyhdr}
\usepackage{wrapfig}
\usepackage{booktabs} % 添加 booktabs 宏包
\usepackage{subcaption}
\pagestyle{fancy}
\fancyhf{}  % 清除所有页眉页脚
\fancyfoot[C]{\thepage}  % 在页脚中央设置页码
\renewcommand{\headrulewidth}{0pt}  % 去掉页眉横线
\renewcommand{\footrulewidth}{0pt}  % 去掉页脚横线

\usepackage{hyperref}
\hypersetup{
    colorlinks=true,
    linkcolor=blue,
    filecolor=magenta,      
    urlcolor=cyan,
    pdftitle={Overleaf Example},
    pdfpagemode=FullScreen,
    }

\usepackage{titlesec}

% 重新定义section格式
\titleformat{\section}
  {\normalfont\Large\bfseries}
  {\chinese{section}、}
  {0pt}
  {}

% 给出常用字号的自定义指令
\newcommand{\chuhao}{\fontsize{42pt}{48pt}\selectfont}
\newcommand{\xiaochu}{\fontsize{36pt}{42pt}\selectfont}
\newcommand{\xiaoer}{\fontsize{18pt}{21.6pt}\selectfont}
\newcommand{\sanhao}{\fontsize{16pt}{20pt}\selectfont}
\newcommand{\xiaosan}{\fontsize{15pt}{18pt}\selectfont}
\newcommand{\sihao}{\fontsize{14pt}{18pt}\selectfont}
\newcommand{\xiaosi}{\fontsize{12pt}{16pt}\selectfont}
\newcommand{\wuhao}{\fontsize{10.5pt}{12.6pt}\selectfont}

% 定义中文数字转换
\titleformat{\section}
  {\normalfont\Large\bfseries}
  {\zhnum{section}、}  % 使用\zhnum而不是\chinese
  {0pt}
  {}

% 标题设置
\title{\xiaochu\textbf{物理实验预习报告}}
\author{}
\date{}

\begin{document}

\vspace*{36pt} % 顶部空白

% 居中填写区域
\begin{center}
    \includegraphics[width = 9cm]{head.png}
    
    \xiaochu\textbf{物理实验预习报告}
    % 预留空白区域
    \vspace{13pt}
    \vspace{13pt}
    \vspace{13pt}
    \vspace{13pt}
    
    \sanhao\textbf{实验名称:\underline{\makebox[11cm]{ 用霍尔法测直流线圈与亥姆霍兹线圈磁场 }}} \\[10pt]
    \sanhao\textbf{实验桌号:\underline{\makebox[8cm]{  }}} \\[10pt]
    \sanhao\textbf{指导教师:\underline{\makebox[8cm]{  }}} \\[20pt]
    
    % 预留空白区域
    \myspace{7}
    
   \sihao\textbf{班级:\underline{\makebox[6cm]{cc98}}} \\[10pt]
    \sihao\textbf{姓名:\underline{\makebox[6cm]{Hydrofoil}}} \\[10pt]
    \sihao\textbf{学号:\underline{\makebox[6cm]{324010}}} \\[30pt]
    
   \sihao\textbf{实验日期: \underline{\makebox[0.8cm]{2025}}年 \underline{\makebox[0.4cm]{11}}月\underline{\makebox[0.4cm]{19}}日  星期\underline{\makebox[0.6cm]{三}}上午}
    
    \vspace{18pt}
    \sihao 浙江大学物理实验教学中心
    
\end{center}

\newpage


    \section{实验综述}
    
    \xiaosi (自述实验现象、实验原理和实验方法,不超过500字,5分)
    
    \subsubsection*{载流圆线圈与亥姆霍兹线圈的磁场:}

    \textbf{(1)载流圆线圈磁场}
    \begin{align*}
        B=\frac{\mu_{0} \cdot N_{0} \cdot I \cdot R^{2}}{2 \cdot(R^{2}+X^{2})^{\frac{3}{2}}}
    \end{align*}
    
    式中\(N_0\)为圆线圈的匝数,X为轴上某一点到圆心$o'$的距离,\(\mu_0=4\pi×10^{-7} H/m\),本实验取\(N_0=400\),\(I=0.400 A\),\(R=0.100 m\),在$O'$处\(X=0\),可计算得:\(B=1.0053×10^{-3} T\)
    
    其磁场的分布图是一条单峰的关于Y轴对称的曲线。
    
    \textbf{(2)亥姆霍兹线圈}

    定义:两个完全相同的圆线圈彼此平行且共轴,通以同方向的电流I,线圈间距等于线圈半径R时,这样的一对线圈称为亥姆霍兹线圈。其磁场分布曲线在两线圈中心连线一段出现一个平台,说明此处为匀强磁场。

    \begin{figure}[htbp]
        \centering
        % 第一张图片
        \begin{subfigure}[b]{0.45\textwidth}
            \centering
            % 替换为你的第一张文件名
            \includegraphics[width=\textwidth]{载流圆线圈的磁场分布.png} 
            \caption{载流圆线圈的磁场分布}
            \label{fig:left}
        \end{subfigure}
        \hfill % 这是一个弹性空格,把两张图撑开
        % 第二张图片
        \begin{subfigure}[b]{0.45\textwidth}
            \centering
            % 替换为你的第二张文件名
            \includegraphics[width=\textwidth]{亥姆霍兹线圈的磁场分布.png}
            \caption{亥姆霍兹线圈的磁场分布}
            \label{fig:right}
        \end{subfigure}
        
        \label{fig:both}
    \end{figure}

    \subsubsection*{ 利用霍尔效应测磁场的原理:}

    \textbf{(1)霍尔效应}:若电流I流过厚度为d的矩形半导体薄片,且磁场B垂直作用于该半导体,由于洛伦兹力的作用,载流子会发生横向偏转,在两横向面a、b之间产生电势差,称为霍尔电势,其方向同时垂直于I与B方向。
    
    \textbf{(2)产生原理}:I通过元件时,空穴有一定漂移速度v,其垂直于B运动时,产生洛伦兹力\(F_B=q·(v×B)\)(q为电子电荷),洛伦兹力使电荷产生横向偏转,偏转的载流子在边界积累,产生横向电场E,直到\(F_E=F_B\),即\(q·E=q·(v×B)\)。
    
    设P型样品的载流子浓度为p,宽度为w,厚度为d,通过样品的电流\(I_H=p·q·v·w·d\),
    则空穴的速度\(v=\frac{I_H}{p·q·w·d}\),代回得\(E=|v×B|=\frac{I_H·B}{p·q·w·d}\),两侧同乘w得
    \begin{align*}
        U_H=E·w=\frac{R_H·I_H·B}{d} \quad (R_H \text{称霍尔系数})
    \end{align*}
    
    应用中一般令\(k_H=\frac{R_H}{d}=\frac{1}{p·q·d}\),称为霍尔元件的灵敏度,灵敏度越大越好。
    
    由上式可知,若已知霍尔片的灵敏度\(k_H\),只要分别测出霍尔电流\(I_H\)和霍尔电势差\(U_H\),即可算出磁场B的大小,此即霍尔效应测场强的原理。

    \subsubsection*{测量载流圆线圈轴线上磁场的分布:}

    (1)正确连接 FB5 型霍尔法亥姆霍兹线圈磁场实验仪与测试架,校准微特斯拉计;

    (2)测试架左边的线圈为固定线圈,固定在 0cm 处,把右边的可动线圈移到合适位置;

    (3)使励磁电流\(I=0.400A\),以圆电流线圈中心为坐标原点,每隔 1.0cm 测一个B值,保持励磁电流不变;

    (4)记录数据,在方格纸上画出\(B-X\)曲线。

    \subsubsection*{测量亥姆霍兹线圈轴线上磁场的分布:}

    (1)移动线圈使两线圈间距\(d=R=10cm\),此时两个圆线圈中心连线的几何中心距测试平面 5cm;

    (2)将两圆电流线圈串联起来,接到磁场测试仪的输出端钮,调节电流输出,使励磁电流\(I=0.400A\),以两个圆线圈中心连线上的中点为坐标原点,每隔 1.0cm 测一个B值;

    (3)记录数据,在方格纸上画出相应的\(B-X\)曲线。

    \subsubsection*{测量载流圆线圈沿 “径向” 的磁场分布:}

    将传感器探头移动到一线圈中心,与线圈平面的夹角为 0°,径向移动探头,每移动 1.0cm 测量一个数据,按正反方向测到 6cm 为止,记录数据,作出磁场分$B-Y$曲线。
   
    
    \section{实验重点}

    \xiaosi(简述本实验的学习重点,不超过100字,3分)


    \begin{enumerate}   
        \item 了解用霍尔效应法测量磁场的原理,掌握 FB511 型霍尔法亥姆霍兹线圈磁场实验仪的使用方法;
        \item 了解载流圆线圈的径向磁场分布情况;
        \item 测量载流圆线圈和亥姆霍兹线圈的轴线上的磁场分布;
        \item 两平行线圈的间距改变为\(d=\frac{1}{2} R\)及\(d=2R\)时,测定其轴线上的磁场分布。

    \end{enumerate}   
    
    
    \section{实验难点}
    
    \xiaosi(简述本实验的实现难点,不超过100字,2分)

    \begin{enumerate}    
        \item 需精准校准微特斯拉计,实验中测试架位置变动后需重新调零;
        \item 霍尔传感器轴向、径向移动需精准控制,每 1cm 测一个数据,操作繁琐;
        \item 磁场易受环境干扰,探头摆放方向偏差会导致测量误差;
        \item 数据处理需绘制多组 B-X 曲线,需保证数据对称性以分析规律。
    \end{enumerate}  \



\newpage

\sihao\textbf{注意事项:}
\xiaosi
\begin{enumerate}
    \item  用 PDF 格式上传“实验报告”,文件名:学生姓名+学号+实验名称+周次。
    \item  “实验报告”必须递交在“学在浙大”的本课程的对应实验项目的“作业”模块内。
    \item  “实验报告”成绩必须在“浙江大学物理实验教学中心网站”-“选课系统”内查询。
    \item 教学评价必须在“浙江大学物理实验教学中心网站”-“选课系统”内进行,学生必须进行教学评价,才能看到实验报告成绩,教学评价必须在本次实验结束后 3 天内进行。
\end{enumerate}

\vspace{1cm}
\xiaosi\centering \textbf{浙江大学物理实验教学中心制}

\end{document}